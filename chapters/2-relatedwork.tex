\chapter{Related Work}\label{chap:relatedwork}
%Give a brief overview of the work relevant for your thesis. 



%\section{Semi-supervised object detection and instance segmentation}
%Semi-supervised learning leverages both unlabeled data and a fraction of labeled data to improve the performance of models. In the context of object detection and semantic segmentation, several recent works have explored various techniques to enhance these tasks.
%
%Most of the recent works in this domain includes a student-teacher model as in  Adaptive Teacher~\cite{Li_2022_CVPR}, Unbiased Teacher~\cite{liu2021unbiasedteachersemisupervisedobject} and Soft Teacher~\cite{xu2021endtoendsemisupervisedobjectdetection}. These methods slightly differ in the type of augmentations(strong, weak and hybrid augmentation) used and the method of action on the pseudo labels. For instance, Unbiased teacher addresses bias in pseudo-labeling by using a teacher-student model. The teacher generates pseudo-labels which are then used to train the student. The process iterates with the student eventually replacing the teacher. Where as the Soft teacher leverages both hard and soft pseudo-labels. The method uses a teacher model to generate soft labels (probabilistic outputs) for unlabeled data, which are used to train a student model.
