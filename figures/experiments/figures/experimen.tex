\usepackage{mwe}
\usepackage{caption}
\usepackage[skip=0.5ex, belowskip=1ex]{subcaption}

\usepackage{tikz}
\usetikzlibrary{quotes,
                tikzmark}


\begin{figure}[htbp]
     \centering
     \setkeys{Gin}{width=0.95\linewidth}
\tikzmarknode{A}{~\begin{subfigure}{0.32\textwidth} % observe ~ before subfigure
     \includegraphics{example-image-empty}
     \caption{caption 1ss}
\end{subfigure}}
\hfill
\begin{subfigure}{0.32\textwidth}
     \includegraphics{example-image-empty}
     \caption{caption 1sg}
\end{subfigure}
\hfill
\begin{subfigure}{0.32\textwidth}
     \includegraphics{example-image-empty}
     \caption{caption 1sd}
\end{subfigure}

 \medskip
\tikzmarknode{B}{~\begin{subfigure}{0.32\textwidth}  % observe ~ before subfigure
     \includegraphics{example-image-empty}
     \caption{caption 1ts}
 \end{subfigure}}
\hfill
 \begin{subfigure}{0.32\textwidth}
     \includegraphics{example-image-empty}
     \caption{caption 1tg}
 \end{subfigure}
\hfill
 \begin{subfigure}{0.32\textwidth}
     \includegraphics{example-image-empty}
     \caption{caption 1td}
\end{subfigure}
\end{figure}

%%%% drawing square bracket
%%%% drawing a line
\begin{tikzpicture}[overlay,remember picture]
    \path[draw=purple, thick, text=purple, scale=1.2]
         ([xshift=10mm] B.south west) -- (B.south west) -- ([yshift=1mm] A.north west) -- ++ (1,0);
\end{tikzpicture}